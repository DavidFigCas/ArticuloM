%&latex

\documentclass[conference]{IEEEtran}
\usepackage[cmex10]{amsmath}
\usepackage{algorithmic}
\usepackage{array}
\usepackage{eqparbox}
\usepackage[english]{babel}
% *** SUBFIGURE PACKAGES ***


\usepackage{graphicx}
\usepackage{epsfig}
\usepackage{epstopdf}
\usepackage{amsmath} 
%\usepackage[caption=false]{caption}
\usepackage[font=footnotesize]{subfig}
\usepackage{stfloats}
\usepackage{float}
\usepackage{subfig}
\usepackage{amssymb}
\usepackage{cite} 
\usepackage[T1]{fontenc}
%\usepackage{hyperref}
\begin{document}

\title{Proyecto de Investigación de cotizador automático por medio de ChatBots}
\author{
\IEEEauthorblockN{D.~Figueroa-Castañeda}\\
\IEEEauthorblockA{Universidad Interamericana~ Maestría en Automatizac,\\
Puebla,~Mexico \\
Email: \{david\_d.figueroa@lainter.edu.mx}
}



% The paper headers
\markboth{}%
{Shell \MakeLowercase{\textit{et al.}}: Bare Demo of IEEEtran.cls for Journals}

% make the title area
\maketitle


 
\begin{abstract}
 In this paper, ... Finally, a simulation ... is shown.
\end{abstract}
\vspace{0.5cm}
\textbf{Keywords}:  Homotopy continuation methods, Path Planning, Mobile  robot, ROS.

\IEEEpeerreviewmaketitle
\section{Introduccion}
\IEEEPARstart{I}n recent decades,... This paper is organized as follows. In Section $\mathbf{II}$, ... In Section $\mathbf{IV}$... Some simulations in Section $\mathbf{VI}$. Finally, the conclusions are presented in Section $\mathbf{VII}$.

\section{Homotopic Continuation Method}

Homotopy continuation method..

\begin{equation}
\label{ohm}
V=I*R:\quad\mathbb{R}^{n}\longrightarrow\mathbb{R}^{n},
\end{equation}

The system:  
 \begin{equation}
    \label{Homotopia_G}
     H(x,\lambda)= \lambda f(x)+ (1-\lambda)(f(x)- f(x_0))=0,
 \end{equation}
 
 \begin{figure}
 	\includegraphics[scale=1]{imagenes/img1.eps}
 	\caption{Imagen de prueba}
 	\label{fig:img1}
 \end{figure}
     
     
     
 where,   $\lambda$ is the homotopy parameter,  $x_0$  is the starting point, $H( x,\lambda) :\mathbb{R}^{n+1}\longrightarrow \mathbb{R}^{n} \text{,} \quad{x} \in\mathbb{R}^{n}$. 
\section{Obstacles}
HPPM uses the...%-----------------------------------------------------------------------------------------------------------

\section{Spherical}
 \begin{figure}[H]
\begin{center}
\includegraphics[width=0.2\textwidth]{imagenes/hiper2.eps} 
\caption{ Seguimiento.}
\label{fig:hiper2}
\end{center}
\end{figure}    


  \begin{figure}[H]
\begin{center}
\includegraphics[width=0.2\textwidth]{imagenes/hiper2.eps} 
\caption{ Seguimiento.}
\label{fig:hiper3}
\end{center}
\end{figure}    
  
\begin{figure}[H]
\begin{center}
\includegraphics[width=0.2\textwidth]{imagenes/hiper2.eps} 
\caption{ Seguimiento.}
\label{fig:hiper4}
\end{center}
\end{figure}    
  
\subsection*{Predictor-Corrector Scheme}
 A proper \cite{plc1} predictor-corrector Figure \ref{fig:hiper4} scheme \cite{Hector1, Gerardo1}...
  
 
 



\section{Experiments}

The efficiency of the \cite{Park-2008} proposed...
\subsection{Successful path for maps with 200 and 2000 obstacles}

We consider two study cases...

  \vspace{-0.1 cm}
\begin{table}[H]
\begin{center}
\resizebox{9cm}{!} {
\begin{tabular}{ |c|c|c|c|c|c|c|c|c| }
\hline
\multicolumn{9}{|c|}{Environment maps}\\
\hline
\multicolumn{1}{|c}{N.Obstacles}&\multicolumn{4}{|c}{200}&\multicolumn{4}{|c|}{2000}\\
\hline
Path&1&2&3&4&1&2&3&4\\
\hline
Steps&919&898&894&999&7165&6404&7406&6953\\
\hline
Time (ms)&504&483 &504&564&41190&38840&48561&39305\\
\hline
Path length&2.10143&2.06822&2.01062&2.2497&2.59544&2.20463&2.57591&2.40284\\
\hline
\end{tabular}
}
\end{center}
\caption{Computation time and length in normalized units for two environment maps.}
\label{table:tiempos}
\end{table}

%%%jashljdhlsjahlfhsd

\section{Conclusions}
 In this work,...

\bibliographystyle{ieeetr}
\bibliography{bibliography}



\end{document}


